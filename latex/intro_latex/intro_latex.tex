\documentclass[12pt]{article}

\usepackage[utf8]{inputenc}
\usepackage[T1]{fontenc}
\usepackage{hyperref}
\usepackage{XCharter}
\usepackage{geometry}
\usepackage{graphicx}
\usepackage{background}
\usepackage{float}

\geometry{
	paper=a4paper,
	top=2.5cm,
	bottom=3cm,
	left=2.5cm,
	right=2.5cm,
	headheight=14pt,
	footskip=1.5cm,
	headsep=1.2cm,
}

\backgroundsetup{
    position = {-2.5, -33},
    scale = 0.35,
    opacity = 0.3,
    angle = 90,
    contents = {
        \includegraphics{assets/pics/logo.pdf}
    },
}

\tolerance=1
\emergencystretch=\maxdimen
\hyphenpenalty=10000
\hbadness=10000
\interfootnotelinepenalty=10000


\begin{document}
\title{A Brief Introduction to \LaTeX}
\author{Allin Cottrell\\\small revamped by Iñaki Amatria Barral}
\date{Universidade da Coruña -- Autumn 2019}
\maketitle

\section{Introduction}
\noindent
\TeX{}, written by Donald Knuth, is a very powerful computer typesetting
program. \LaTeX{}, originally written by Leslie Lamport, is a large set of
macros (essentially, shortcuts) for \TeX{}. Most people use \LaTeX{}
rather than ``plain'' \TeX{}. If you want to know, \TeX{} is pronounced
``tech'' rather than ``tex''.

Typesetting with \LaTeX{} can get as sophisticated as you like---it can be
used to produce printed copy that is better than most books being
published these days---but we will stick to the basics here.

The basic idea with \TeX{} and \LaTeX{} is that you start with a plain text
file containing the text you want to typeset. You indicate the way you want
the typesetting done by means of certain special codes, most of which begin
with a backlash ({\tt\char`\\}). When the text file is ready, you compile it
using the \TeX{} processor, then preview the product on screen. If it is OK
you can go ahead and print it, otherwise go back to the editing stage,
re-compile the file, preview it again, and so on.

Superficially, \LaTeX{} looks more difficult to use than a standard word
processor, because it is not WYSIWYG (``What you see is what you get'').
As you are typing your basic text file with the typesetting commands
included, you do not see what it will look like after compilation. But
after a little practice you will find this is not really a problem. Also,
when you get to the preview stage, ``what you see'' is a considerably more
accurate representation of ``what you'll get'' at the printer than with
ordinary word processors. The great advantages of \LaTeX{} over word
processors are:

\begin{itemize}
\item You don't have to fuss with the detailed appearance of your
document. You concentrate on (a) getting the text right and (b) the
overall logical structure of the piece. \TeX{} takes care of the detailed
formatting.
\item The support for the typesetting of mathematics is unrivaled.
\item Input files for \TeX{} are entirely portable. If I have a document
in the form of a \LaTeX{} file I can email it to anyone who has access to
the \TeX{} processor---wheter on the PC, the Mac, or a Unix system---and
they will be able to reproduce the same printed document.
\end{itemize}
\section{Document skeleton}
\noindent
The skeleton of the \LaTeX{} document looks like this

\begin{figure}[ht]
\centering
\begin{tabular}{l}
\hline
{\tt\char`\\documentclass\char`\{article\char`\}}\\
\\
{\tt\char`\\begin\char`\{document\char`\}}\\
{\tt Hello World!}\\
{\tt This is my very first \char`\\LaTeX\char`\{\char`\}{} document.}\\
{\tt\char`\\end\char`\{document\char`\}}\\
\hline
\end{tabular}
\caption{A minimal \LaTeX{} file.}
\end{figure}

The first line specifies the type or class of the document, in this case
an article. This is the class you are mostly likely to use (other classes
include report, book, letter and beamer). The default print size for an
article is 10-point type. If you find it a bit too small you can use the
11-point (or even the 12-point) option:

\vspace{\baselineskip}
{\tt\char`\\documentclass[11pt]\char`\{article\char`\}}
\vspace{\baselineskip}

Beyond that, the {\tt\char`\\begin\char`\{document\char`\}},
{\tt\char`\\end\char`\{document\char`\}} pair constitutes the minimal
set of special commands for a ``valid'' \LaTeX{} document. In between
these commands goes your text.

Moving beyond the minimal involves learning a little about these topics:

\begin{itemize}
\item Document-structuring commands
\item Typeface-changing commands
\item Special characters
\item \LaTeX{} math mode
\end{itemize}

We will examine these in turn.
\section{Document-structuring commands}
\subsection{Lines and paragraphs}
\noindent
You needn't pay much attention to the line breaks in your source file;
\LaTeX{} will make its own decisions on where to break lines in order to
produce properly justified text. The only important thing to remember is
that if you want a paragraph break, insert a blank line, or in other words
hit the Enter key twice in your source file.
\subsection{Author, title, date}
\noindent
The title and author's name can be inserted at the beginning of the
document thus

\begin{figure}[H]
\centering
\begin{tabular}{l}
\hline
{\tt\char`\\title\char`\{Pride and Prejudice\char`\}}\\
{\tt\char`\\author\char`\{Jane Austen\char`\}}\\
{\tt\char`\\maketitle}\\
\hline
\end{tabular}
\caption{Author, title declaration.}
\end{figure}

Coding the elements in this way will automatically put them into a
centered format, using a larger font than the basic text. The {\tt
\char`\\maketitle} command also inserts the date on which the
document is compiled. Since this is inappropriate for {\it Pride and
Prejudice} it can be overridden by specifying a date before {\tt
\char`\\ maketitle}.

\vspace{\baselineskip}
{\tt\char`\\ date\char`\{1813\char`\}}

\subsection{Sectioning}
\noindent
Couldn't be easier: Just type \texttt{\char`\\ section\char`\{}\emph{text}%
{\tt\char`\}}, replacing \emph{text} with the specified section-heading you
want to use. The section-heading text will be automatically formatted, put
in the boldface type, and given a number. The numbers will be recalculated
if you move text around. For instace, the section-heading above was produced
using

\vspace{\baselineskip}
\texttt{\char`\\ section\char`\{Doucment-structuring commands\char`\}}
\vspace{\baselineskip}

\noindent
Variations: If you want an unnumbered section heading, do

\vspace{\baselineskip}
\texttt{\char`\\ section*\char`\{Example of unnumbered heading\char`\}}
\vspace{\baselineskip}

That is, adding a {\tt *} turns off the numbering. If you want a subsection
heading, use {\tt\char`\\subsection\char`\{\char`\}}. You turn off the
numbering for subsections in the same way.
\subsection{Footnotes}
\noindent
To insert a footnote, just type (in the text itself, where you want the
footnote-maker to appear) {\tt\char`\\footnote\char`\{}\emph{text}{\tt%
\char`\}}, where \emph{text} is replaced by the text of the footnote. The
notes will be typeset at the foot of the page in a smaller type-size, and
will be automatically numbered.
\subsection{Itemized lists}
\noindent
To get a bulleted list, just do

\begin{figure}[H]
\centering
\begin{tabular}{l}
\hline
{\tt\char`\\begin\char`\{itemize\char`\}}\\
{\tt\char`\\item This is a first item.}\\
{\tt\char`\\item And this is a second one.}\\
{\tt\char`\\item And so on...}\\
{\tt\char`\\end\char`\{itemize\char`\}}\\
\hline
\end{tabular}
\caption{A simple bulleted list.}
\end{figure}

To have the list enumerated rather than bulleted, use {\tt\char`\\begin%
\char`\{enumerate\char`\}} and {\tt\char`\\ end\char`\{enumerate\char`\}}
instead. The numbering will be taken care of automatically.
\section{Typeface-changing commands}
\noindent
You will need less of these than with a standard word processor, because
\LaTeX{} is smart enough to change the typeface itself, in a consistent and
pleasing manner, for section-heading, footnotes and other special features
of the text. But sometimes you want italics for a book-title or for
emphasis, boldface for emphasis or definitions, or whatever. Here is how
to do it

\begin{figure}[ht]
\centering
\begin{tabular}{l l}
This {\it word} is in italics. & {\tt This \char`\{\char`\\it word%
\char`\}{} is in italics}.\\
{\bf These words} in boldface. & {\tt\char`\{\char`\\ bf These words%
\char`\}{} in boldface}.\\
Here's one in {\tt typewriter}. & {\tt Here's one in \char`\{\char`\\tt%
typewriter\char`\}.}\\
And some {\sc Small Caps}. & {\tt And some \char`\{\char`\\sc Small Caps%
\char`\}.}\\
\end{tabular}
\caption{Simple typeface-changing commands.}
\end{figure}

That is, put braces {\tt`\char`\{'} and {\tt`\char`\}'} around the text the
typeface of which you wish to change, and insert the readily memorable code
for the type-style you want.

\section{Special characters}
\noindent
The issue of special characters (outside of the math mode, which is
discussed below) arises two instances:

\begin{itemize}
\item When you want a character that is not in the {\tt ASCII}
character set (and not on the computer keyboard).
\item When you want to use in your text a character that has a
special meaning to \TeX.
\end{itemize}

Take the non-{\tt ASCII} characters first. You can consult one of the
standard \LaTeX{} references (at the end of this document) for a full
listing, but here are a few of the most common ones:

\begin{itemize}
\item Left-hand curly double-quote (``): type two accent graves or whatever
you want to call them. Type just one of these for a single left-hand quote
(`).
\item Right-hand curly double-quote (''): type two single straight quotes.
Type just one for a single right-hand quote (').
\item Endash: This is properly used in place of a hyphen in giving number
ranges (pp. 121--34; the war of 1914--18). Type two hyphens in succession
({\tt 1914--18}).
\item Emdash: The proper ``dash'' to use in punctuating text---like this.
Type three hyphens in succession.
\end{itemize}

Getting the above-mentioned symbols right is a matter of good style. But
dealing correctly with the symbols that have a special meaning to \TeX{}
is essential if your life is to compile correctly. Watch out for the
following in particular:

\begin{itemize}
\item The dollar sign ({\tt\$}): This is used as the delimiter for
\LaTeX{} math mode (see below). If you want a regular dollar sign in your
text (like this: \$9.99) you must prefix it with a backslash ({\tt%
like this: \char`\\\$9.99}).
\item The percent sign ({\tt\%}): This is used as the comment character in
\TeX{}: Anything following this sign on a given line of a \TeX{} input
file is ignored by the processor. If you want one in your text, use {\tt%
\char`\\\%}.
\item The ampersand ({\tt\&}): This is as the tab character in \TeX{}:
Type {\tt \char`\\\&} if you want an ampersand to appear in the text.
\end{itemize}
\section{\LaTeX{} math mode}
\noindent
As mentioned above, the ability to typeset math accurately is one of the
principal attractions of \TeX{} and \LaTeX{}. To get started at this, you
need to know the way into the two math modes. To get into inline math
mode, type a dollar sign, {\tt\$}. Type another one to exit. By the
``inline math'' I mean mathematical symbols that appear as part of the
regular text. To get into display math mode (for equations that are a set
off on lines by themselves), type {\tt \$\$} (or {\tt\char`\\[}). To
terminate this mode, close with {\tt\$\$} again, or {\tt \char`\\]} if
you started with {\tt \char`\\ [}. One other variant should be mentioned:
if you want an automatically numbered equation, enter math mode with
{\tt\char`\\begin\char`\{equation\char`\}} and exit with {\tt\char`\\%
end\char`\{equation\char`\}}.

Once in math mode, you don't have to worry about the spacing of what you
type; \TeX{} will take care of that. You do have to remember some fairly
simple codes for symbols that do not appear on the keyboard, and for
special mathematical ways of arranging symbols. The standard refenreces
will give you all the details. For now, you can learn some of basics by
example

\begin{figure}[ht]
\centering
\begin{tabular}{l l}
$y = f(x)$ & {\tt\$y = f(x)\$}\\
$y_t = \alpha + \beta x_t + \epsilon_t$ & {\tt\$y\char`\_t = \char`\\%
alpha + \char`\\beta x\char`\_t + \char`\\epsilon\char`\_t}\\
$\frac{1}{3} = 0.333\ldots$ & {\tt\$\char`\\frac\char`\{1\char`\}%
\char`\{3\char`\}{} = 0.333\char`\\ldots\$}\\
$8 \times 9 = 72$ & {\tt\$8 \char`\\times 9 = 72\$}\\
$y = x^2$ & {\tt\$y = x\char`\^2\$}\\
$p \vee F \equiv p$ & {\tt\$p \char`\\vee F \char`\\equiv p\$}\\
\end{tabular}
\caption{Some basic math formulas.}
\end{figure}

As you can see, in math mode the Greek letters are obtained simply by
spelling them out in English, preceded by a backslash. To get the
uppercase versions, just capitalize the first letter (for instace {\tt\$%
\char`\\Omega\$} gives $\Omega$).

Subscripts and superscripts are obtained in math mode by means of
{\tt\char`\_} and {\tt\char`\^} respectively. For complex subscripts and
superscripts, enclose the argument in braces (for instansce {\tt\$K%
\char`\^\char`\{\char`\\alpha + \char`\\beta\char`\}\$} gives
$K^{\alpha + \beta}$).
\section{Files and filenames}
\noindent
At this point it may also be useful to learn about the file-naming
conventions in the \TeX{} world. Your input file (text plus formatting
commands) is typically given a {\tt .tex} extension (for instance
{\tt my\char`\_file.tex}). The \TeX{} processor takes this as input and
outputs a {\tt .pdf} file (e.g. {\tt my\char`\_file.pdf}). In the process
of creating a {\tt pdf} file, the processor also writes a {\tt log} file
and an {\tt aux} file. Generally you won't be much concerned with these,
but if things go wrong the log file may contain some useful information.
\section{About the software}
\noindent
Although these pieces of software may be bundled together in a package, you
need three distincts programs to constitute a fully functional \LaTeX{}
system.

\begin{enumerate}
\item A suitable editor for preparing and modifying your input files.
\item The \TeX{} processor itself.
\item A program to display the compiled version of your file on screen.
\end{enumerate}

The wonderful thing about all this software is that is in the public
domain. You can download it for free from any of many archive sites
maintained at both academic institutions and commercial Internet servers
around the world.
\section*{Further reading}
\noindent
The basic source on \LaTeX{} is Leslie Lamport's {\it \LaTeX{}, A Document
Preparation System}, Second Edition, Addison-Wesley, 1994. Another useful
reference is Gooseens {\it et al.}, {\it The \LaTeX{} Companion},
Addison-Wesley, 1994. For anyone who gets an urge to do some serious
\TeX{} hacking (you can customize \emph{any} aspect of \TeX{}'s operation
if you're willing to learn a bit about the language), the essential
reference is Donald Knuth, {\it The \TeX{}Book}, also Addison-Wesly 1994.
\end{document}
